\documentclass[english]{jsarticle}
\usepackage{amsmath,booktabs}
\begin{document}

\section{Model Extraction}
Model $M$ of the original contract. We take the HTLC contract which is written in Solidity as an example.
Solidity consists of one or more internal state variables (denoted as $v\in V$) and external methods ($f \in F$) that changes the state.

An invocation of a method $f(\bar{x})$ is typically implemented as follows.
It first checks guard conditions on $V$ and $\bar{x}$ of the method.
If they are not satisfied, method body will be executed. Otherwise, the invocation would be reverted.
Thus state variable of $M$ can be extracted from the conditions.

Here shows the target HTLC contract. It has three main methods: \textit{newContract}, \textit{withdraw}, and \textit{refund}.
For example, \textit{withdraw} method has four guard conditions:

\begin{verbatim}
haveContract(_contractId)
contracts[_contractId].receiver == msg.sender
ontracts[_contractId].withdrawn == false
contracts[_contractId].timelock > now
\end{verbatim}

Since the first condition performs access control, which can be ignored in transpilation process.
We can extract atomic propositions in $M$ from the second and third conditions,
where \texttt{locked == 1} iff .

\section{Transpilation of Program}
This section provides transpilation of the original program $P$ and the enforced model $M`$ into a transpiled program $P'$.
(Here we assume that the program consists of Ethereum contracts written in Solidity and client program in Javascript.)

In our example, $M'$ is a composition of two HTLC contracts (each for Security and Cash respectively).
Thus we need to specialize the original HTLC contract for .
As described in section ??, an action $a$ in the enforced HTLC composite model $M'$ are $C.f$, where $C$ is, which is either \texttt{Sec} or \texttt{Cash}, a name of the ledger, and $f$ is a method name.

\subsection{Transpilation of Contract}
Here we sketch transpilation the HTLC contract for the ledger $C_0$. 
We need to add a state variable \texttt{state} to the contract.
We also add to each method a pre- and post-condition for the \texttt{state}.
In Solidity it can be implemented using \textit{modifier} mechanism.
We take an transition $q_i \xrightarrow{C.f} q_{i+1}$ of $M'$ (which means that the HTLC contract has a method $f()$), which is transpiled as follows.

(1) If $C \neq C_0$, add a method \texttt{function $C\_f$()\{\}} to the transpiled contract. Its pre-condition would be \texttt{state == $q_i$} and post-condition would be \texttt{state == $q_{i+1}$}. This method just change the contract's state and does nothing else.

(2) If $C=C_0$, there are two subcases.

(a) If the execution of $f$ branches, i.e., there is two following events on $C.f$; $q_i \xrightarrow{C.f} q_{i+1} \xrightarrow{C.f\_{end}} q_{i+2}$ and $q_i \xrightarrow{C.f} q_{i+1} \xrightarrow{C.f\_{err}} q'_{i+2}$,
the original method $f$ is converted to $C\_f()$.
$C_f()$ first checks state pre-conditon and change the current state to $q_{i+1}$, then executes the body of $f()$.
If it succeeds, it emits an event $C.f\_{end}$ and change its state to $q_{i+2}$.
If the execution fails, it emits an event $C.f\_{err}$ and change its state to $q'_{i+2}$.

(b) If there's no such events, we simply transpile $f()$ of the contract into $C\_f()$, whose pre-condition on \texttt{state} is $q_i$ and post-condition is $q_{i+1}$.

\subsection{Transpilation of Client code}
Transpilation of the client code is similar. Note that $M'$ is specialized into two contracts on the separated ledgers.
Thus the transpiled client need to synchronize states between both contracts.

Assume that the original code includes TX invocation $C.f()$, where $C$ is a ledger (or contract) name.
We transpile it to invocations of methods the transpiled contracts $C.C\_f(); C'.C\_f()$, where $C'$ is a name of the other ledger than $C$.
If the action corresponds to the case (2b), that completes the transpilation.
If it's the case for (2a), additional event handling is needed. The invocation of $C.C\_f()$ emits an event $ev$ which is either $C.f\_{end}$ or $C.f\_{err}$.
The transpiled client should subscribe the event and invoke the method of $C'$ which corresponds to it.
For example, if it received $C.f\_{end}$, it should call $C'.C\_f\_{end}()$.

\section{Evaluation}

\begin{figure}
\centering
\begin{tabular}{llrlrlr}\toprule
& \multicolumn{2}{c}{A. Original}
& \multicolumn{2}{c}{B. Wrapper}
& \multicolumn{2}{c}{C. Transpiled} \\
\cmidrule(lr){2-3} \cmidrule(lr){4-5} \cmidrule(lr){6-7}
Testcase & Result & Gas & Result & Gas & Result & Gas \\
\midrule
(1) Normal (both withdraw) & success & $450,928$ & success & $450,928$ & success & $661,473$ \\
(2) Normal (both refund) & success & $404,354$ & success & $404,354$ & success & $695,941$ \\
(3) Abnormal (Non-termination) & not detected & $372,568$ & detected & $372,440$ & detected & $664,855$ \\
(4) Abnormal (Invalid state) & not detected & $427,641$ & detected & $372,568$ & detected & $661,775$ \\
(5) Abnormal (TX. disorder) & error & $294,080$ & detected & $294,080$ & detected & $362,412$ \\
\bottomrule
\end{tabular}
\caption{Evaluation result.}
\end{figure}

(Maybe description about each implementation somewhere?)

We provide five testcases.
In the testcase (1), both Seller and Buyer successfully withdraw assets. 
Both B and C implementations go through valid states ($q_2\to q_4\to q_5\to q_6 \to q_8\to q_{10}\to q_{12}$).
But the C consumes more gas due to overhead of implementing state machines in contracts.
Similarly, testcase (2) runs without errors by all implementation ($q_2\to q_4\to q_5\to q_6 \to q_7 \to q_9\to q_{11}\to q_{13} \to q_{12}$).

In testcase (3), Buyer fails to withdraw and terminate the protocol. Implementation B and C detects that protocol terminates 
errornous state, which A cannot detect. (TBW: how to write?)

In testcase (4), Buyer fails to withdraw and then Seller try to refund the security.
It would result in the situation where both security and cash belong to the Seller.
While implementation A pass the refund TX, B and C properly reject the refund.
Because when the withdraw fails, the composite state machine is at $q_{12}'$.
At the state \texttt{Sec.refund} transition is not allowed.


In testcase (5), \texttt{Cash.refund} is called right after \texttt{newContract} transactions.
In this case, TX invocation in A results in an error since it does not reach timeout.
On the other hand, B and C rejects to invoke \texttt{refund} due to they are at an invalid state to call the transaction.
\end{document}